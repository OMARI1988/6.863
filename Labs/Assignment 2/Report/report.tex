%%%%%%%%%%%%%%%%%%%%%%%%%%%%%%%%%%%%%%%%%%%%%%%%%%%%%%%%%%%%%%%%%%%%%%%%%%%%%%%%%%%%%%
% Answers for 6.863 Assignment2
% Created by Olga Wichrowska 06/2010
% Modified by rcb 9/1/12
%%%%%%%%%%%%%%%%%%%%%%%%%%%%%%%%%%%%%%%%%%%%%%%%%%%%%%%%%%%%%%%%%%%%%%%%%%%%%%%%%%%%%%
\documentclass[10pt]{article}

\setlength{\topmargin}{-0.8in}
\setlength{\oddsidemargin}{0in}
\setlength{\textwidth}{6.5in}
\setlength{\textheight}{9.25in}

\usepackage{sectsty}
\usepackage{amsmath}
\usepackage{amsfonts}
\usepackage{amssymb}
\usepackage{graphicx}
\usepackage{listings}
\usepackage{subfigure}

\sectionfont{\large}
\subsectionfont{\normalsize}
\subsubsectionfont{\small}

\newcommand{\handout}[3]{
  \renewcommand{\thepage}{#1 - \arabic{page}}
  \noindent
  \begin{center}
    \hspace*{-0.25in}\framebox[6.5in]{
      \vbox{
        \hbox to 6.25in { {\bf 6.863J/9.611J: Natural Language Processing } 
                          \hfill Prof. Robert C. Berwick }
        \vspace{4mm}
        \hbox to 6.25in { {\Large \hfill #1  \hfill} }
        \vspace{2mm}
        \hbox to 6.25in { {\it #2 \hfill #3} }
        }
      }
  \end{center}
  \vspace*{4mm}
}

\begin{document}
\handout{Assignment 2: Context-free grammar writing}{My posse consists of: [COLLABORATORS]}{Salman Ahmad (saahmad@mit.edu)}

\begin{enumerate}

\item {\bf Hand in the output of a typical sample run of 10
  sentences from your random sentence generator.  Be sure to attach
  the code so we can test it.}

{\tt

is it true that the president kissed a chief of staff on a pickle on every chief of staff under a pickle in a pickle with the sandwich with every chief of staff on every floor ?

a president under every chief of staff understood a pickle .

a president on a delicious floor under the president under a sandwich under every pickle with every sandwich on every chief of staff under a president on the sandwich on the delicious pickle on every president on the pickle on a sandwich on a pickle under a chief of staff with a fine president under the chief of staff under the chief of staff on a president in every floor under the pickle in a delicious president with the pickle under a floor on a sandwich with a chief of staff on a sandwich in the chief of staff in the pickle under every floor with the pickle under every sandwich on the pickled pickled president in every sandwich in a sandwich with every floor on a perplexed sandwich on a floor on the chief of staff under a fine floor on a sandwich with every sandwich under every floor in the chief of staff under a floor in every pickle on a pickled delicious president under every perplexed chief of staff in the floor in a pickle with every pickle with the chief of staff on the chief of staff with a chief of staff with a floor under the floor in every floor with the floor with every chief of staff on every pickle in a chief of staff on every pickle under a chief of staff with a pickle on the president on a sandwich in the pickle with a pickle in a floor in a fine chief of staff under a sandwich under a pickle under the chief of staff under every chief of staff on every pickle with every president with the president on the sandwich in the president in the chief of staff in every president under a president on the pickle with the pickle in every fine president under a pickle on a pickled president with a sandwich on a president on the pickle with the fine pickle under the sandwich in the floor on every floor with every floor with every delicious floor in a sandwich under every chief of staff under every pickle under a sandwich in a pickle in every floor in every perplexed sandwich in a floor in every president on the pickle on every pickle in every chief of staff under a chief of staff in a sandwich under a fine sandwich with a pickle on every floor on a pickle under the delicious floor in every president under the floor under a president under the sandwich on every pickle with a chief of staff on the chief of staff in every president under the sandwich under the president under a sandwich with a sandwich on the president under the fine sandwich under a floor on a chief of staff in every sandwich with every chief of staff under every chief of staff with the perplexed sandwich with a sandwich under the fine president with every chief of staff on the floor on the chief of staff in the fine floor on every pickle under a sandwich under every president on every floor under the pickle in every floor in every pickled pickle on every president in every pickle on a president under every sandwich under every floor under every chief of staff with every chief of staff on every president under the fine floor under every chief of staff in the sandwich with every delicious floor under the president with a floor in every chief of staff on every sandwich with a president with a pickle in every president in the chief of staff on the chief of staff with a chief of staff under the sandwich under the pickled fine chief of staff on the pickle with the sandwich in a chief of staff pickled the pickle !
every president pickled a floor !

the floor ate the chief of staff !

every chief of staff in the pickle with every pickle in a perplexed sandwich on a pickle under every sandwich ate every pickle in the chief of staff under the sandwich on every floor in a sandwich with every president with every president with every chief of staff on the chief of staff in the floor in a sandwich under every sandwich on a floor on every sandwich on every delicious fine pickle on a floor in the president !
a floor ate every floor under every sandwich under every president with the pickle in the sandwich under the perplexed sandwich under the delicious delicious fine chief of staff !

is it true that every president kissed every pickle ?

every floor on every sandwich in the sandwich pickled a sandwich under the floor in the president in a sandwich with the pickled sandwich in every chief of staff on the president in a perplexed pickle in the president on every president on every floor with the floor .

the sandwich ate a president .

}

\item\begin{enumerate}
\item {\bf Why does your program generate so many long sentences?}
    Specifically, what grammar rule is responsible and why? What is
    special about this rule?

Long sentences are generated when you have a rule that (or s set of rules) that form a recessive loop. That is a rule $A$ that either directly calls $A$ itself or calls another rule that eventually calls $A$. In our grammar, one such rule is:

{\tt NP $\rightarrow$ NP  PP}

The other things that are special about this rule is that there are only two NP rules. These means that 50\% of the time we will be performing a recursive loop.

\item The grammar allows multiple adjectives, as in, \verb|the fine perplexed pickle|. 
{\bf Why do your generated program's sentences exhibit this so rarely?}

The rule that allows multiple adjectives is:

{\tt Noun $\rightarrow$ Adj Noun}

The important thing about this rule is that there are many Noun alternatives which means that it will be recusing at a lower proability.

\item {\bf Which numbers must you modify to fix the problems in 2(a) and
    2(b), making the sentences shorter and the adjectives more
    frequent? (Check your answer by running your new generator and
    show that they work.)}

\item {\bf What other numeric adjustments can you make to the grammar in
  order to favor more natural sets of sentences? Experiment. Hand in
  your grammar file as a file named \verb+grammar2+, with comments that
  motivate your changes, together with 10 sentences generated by the
  grammar.}

\end{enumerate}
\item {\bf Modify the grammar into a new single grammar that can also generate the types of
  phenomena illustrated in the following sentences.}
\begin{enumerate}
\item \verb| Sally ate a sandwich .|
\item \verb| Sally and the president wanted and ate a sandwich .|
\item \verb| the president sighed .|
\item \verb| the president thought that a sandwich sighed .|
\item \verb| that a sandwich ate Sally perplexed the president .|
\item \verb| the very very very perplexed president ate a sandwich .|
\item \verb| the president worked on every proposal on the desk .|
\end{enumerate}

\noindent
{\bf Briefly discuss your modifications to the grammar. Hand
  in the new grammar (commented) as a file named \verb|grammar3| and about 10
  random sentences that illustrate your modifications.}

\item {\bf Give your program an option ``\verb|-t|'' that makes it produce
  trees instead of strings. Generate about 5 more random
sentences, in tree format. Submit them as well as the commented code
for your program.} 

\item When I ran my sentence generator on \verb|grammar|, it produced
  the sentence:
\begin{verbatim}
every sandwich with a pickle on the floor wanted a president .
\end{verbatim}
\noindent
This sentence is ambiguous according to the grammar, because it could
have been derived in either of two ways.
\begin{enumerate}
\item  One derivation is as follows; {\bf what is the other one?}

\bigskip
\begin{verbatim}





    (START (ROOT (S (NP (NP (NP (Det every)
                                (Noun sandwich))
                                     (PP (Prep with)
                                         (NP (Det a)
                                             (Noun pickle))))
                                (PP (Prep on)
                                    (NP (Det the)
                                        (Noun floor))))
                             (VP (Verb wanted)
                                 (NP (Det a)
                                     (Noun president))))
                  .))
\end{verbatim}
\item {\bf Is there any reason to care which derivation was used?}
\end{enumerate}

\item\begin{enumerate}
 \item {\bf Does the parser
  always recover the original derivation that was ``intended'' by
  \verb|randsent|? Or does it ever ``misunderstand'' by finding an alternative
  derivation instead?  Discuss. (This is the only part of question 6a
  that you need to answer.)}

\item {\bf How many ways are there to analyze the following Noun Phrase (NP)
  under the original grammar?  Explain your answer.}

\item By mixing and matching the commands above, generate a bunch of
  sentences from \verb|grammar|, and find out how many different parses they
  have. Some sentences will have more parses than others. {\bf Do you
  notice any patterns? Try the same exercise with \verb|grammar3|.}

\begin{enumerate} 
\item {\bf Probability analysis of first sentence: Why is \verb|p(best_parse)| so small?  What probabilities were
  multiplied together to get its value of 5.144032922e-05?}\\
\noindent
\verb|p(sentence)| is the probability that \verb|randsent| would
  generate this sentence. {\bf Why is it equal to \verb|p(best_parse)|?}\\
\noindent
{\bf Why is \verb+p(best_parse|sentence)+=1?}

\item {\bf Probability analysis of the second sentence:  \\
\noindent
What does it mean that \verb+p(best_parse|sentence)+ is 0.5 in this
case? \\
\noindent
Why would it be {\it exactly} 0.5?}

\item {\bf Cross-entropy of the two sentence corpus. Explain exactly
    how the following numbers below were calculated from the two sets
    of numbers above, that is, from the parse of each of the two
    sentences.}

\item {\bf Based on the above numbers, what {\it perplexity} per word did
  the grammar achieve on this corpus?}

\item {\bf The compression program might not be able to compress the
    following corpus that consists of just two sentences very well.
    Why not? What cross-entropy does the grammar achieve this time?
    Try it and explain.}  (The new 2 sentence corpus is given below.)

\item {\bf How well does {\tt grammar2} do on average at predicting
    word sequences that it generated itself?  Please provide an answer
    in bits per word.  State the command (a Unix pipe) that you used
    to compute your answer.}

\item If you generate a corpus from {\tt grammar2}, then {\tt
    grammar2} should on average predict this corpus better than {\tt
    grammar} or {\tt grammar3} would. In other words, the entropy will
  be {\it lower} than the cross-entropies. {\bf Check whether this is true:
    compute the numbers and discuss.} \end{enumerate}
\end{enumerate}

\item  {\bf Think about all of the following phenomena, and extend your grammar
  from question 3 to handle them. (Be sure to handle the particular
  examples suggested.)  Call your resulting grammar
  \verb|grammar4| and be sure to include it in your write-up along with examples of it
  in action on new sentences like the ones illustrated below.}

\begin{enumerate}
\item {\it Yes-no questions}.

\item {\it WH-word questions.} 

\end{enumerate}
\end{enumerate}
\end{document}

